\section{Rapportmall}
\label{sec:rapportmall}

\subsection{Inledning}
\label{subsec:inledning}

Detta är en rapportmall med diverse exempel. Den är ständigt under
utveckling. Om du vet att du kan göra bättre - gör det och visa mig
sedan, målet med mallen är nämligen att lära mig \LaTeX. Jag hoppas
givetvis också att jag i processen lyckas skapa en rapportmall som
fungerar, ser bra ut i allas ögon och ger mig makt och status.

Information om vad den ska \emph{innehålla} finner du på \url{http://www.fy.chalmers.se/edu/lab/pdf/rapportskrivning.pdf}

Nedan infogar jag en riktig slask-text, så att du kan se exempel på
hur man löser vissa saker. Om du saknar någonting så skicka ett mail
så ska jag försöka lösa problemet.

Mallen avänder sig av klassen \texttt{report} samt en massa paket. Du
kommer att behöva paketen \texttt{booktabs}, \texttt{qtree},
\texttt{fncychap} samt \texttt{epigraph}. \texttt{booktabs} behövs för
tabellerna. Det ger t.ex. snyggare linjer. \texttt{qtree} skapar fina
träd. \texttt{fncychap} och \texttt{epigraph} används egentligen inte, men om du vill ha
fina kapitel så kan du använda dem.

\subsection{\texttt{martin.cls}}
\label{subsec:paketet}

Detta är mitt eget paket. Det innehåller en del småsaker som kan vara
nyttiga. Många av dessa är direkt hämtade från
\url{http://www.dd.chalmers.se/latex}. Ett par exempel finns nedan.

En fullständig dokumentation av paketet finns inte än.